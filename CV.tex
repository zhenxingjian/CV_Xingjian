% LaTeX resume using res.cls
\documentclass[margin]{res}
%\usepackage{helvetica} % uses helvetica postscript font (download helvetica.sty)
%\usepackage{newcent}   % uses new century schoolbook postscript font
\setlength{\textwidth}{5.1in} % set width of text portion
% \usepackage [colorlinks,linkcolor=blue,anchorcolor=blue,citecolor=green]{hyperref}
\renewcommand{\baselinestretch}{1.2} 
\usepackage{url}
\usepackage{enumitem}
\usepackage{hyperref}
\usepackage{xcolor}
% \usepackage{setspace}
% \usepackage{parskip} % This package makes the spacing to be wider. -Qi
\begin{document}


% Center the name over the entire width of resume:
 \moveleft.5\hoffset\centerline{\large\bf Xingjian Zhen}
% Draw a horizontal line the whole width of resume:
 \moveleft\hoffset\vbox{\hrule width\resumewidth height 0.1pt}\smallskip
% address begins here
% Again, the address lines must be centered over entire width of resume:
 \moveleft.5\hoffset\centerline{Department of Computer Science, University of Wisconsin-Madison}
 \moveleft.5\hoffset\centerline{5770B Medical Science Center. Madison, WI, 53705}
 \moveleft.5\hoffset\centerline{\href{mailto:xzhen3@wisc.edu}{xzhen3@wisc.edu}}
 \moveleft.5\hoffset\centerline{\href{https://zhenxingjian.github.io/homepage/}{\textcolor{blue}{[Homepage]\footnote{\url{https://zhenxingjian.github.io/homepage/}}}} \href{https://github.com/zhenxingjian}{\textcolor{blue}{[Github]\footnote{\url{https://github.com/zhenxingjian}}}}}

\begin{resume}
\section{RESEARCH \\ INTERESTS}
                My research interest is about structured data or high-order features in Computer Vision. Since it's quite mature for Euclidean space deep learning, I would like to explore some different structured data or constrained data. The covariance matrices of the feature vectors can be viewed as the second-order features. Those high-order features are constrained by their own structure, i.e., Symmetric Positive Definite. Thus this will break the basic rules in Euclidean space. But those matrices are more robust to noise and contain more high-level information. I want to try some neural networks in Euclidean space, and extend them to those structured data.

\section{EDUCATION} 
                {\sl {Ph.D. Student}}, Department of Computer Science \hfill 2017 - 2022(Expected)\\
                {\bf University of Wisconsin-Madison}, WI, U.S. \\
                {\sl {B.E.}}, Department of Electronic Engineering \hfill 2013 - 2017\\
                {\bf Tsinghua University}, Beijing, P.R. China. 
                
\section{Paper}
                \begin{itemize}\itemsep -2.2pt
                \item \textbf{[\emph{In submission}]} ``ManifoldGLOW: Extending Flow-based Generative Models to Manifolds.''\\
                            Liu Yang, \textbf{Xingjian Zhen}, Rudrasis Chakraborty, Vikas Singh.   
                \item \textbf{[\emph{CVPR, 2020}]} ``CPR-GCN: Conditional Partial-Residual Graph Convolutional Network in Automated Anatomical Labeling of Coronary Arteries.''\\
                            \textbf{Xingjian Zhen}, Han Yang, Ying Chi, Lei Zhang, Xiansheng Hua.   
                \item \textbf{[\emph{ICCV, 2019}]} ``Dilated Convolutional Neural Networks for Sequential Manifold-valued Data.'' \\
                             \textbf{Xingjian Zhen}, Rudrasis Chakraborty, Nicholas Vogt, Barbara B. Bendlin, Vikas Singh. 
                \item \textbf{[\emph{AAIC, 2019}]} ``Sequential Deep Learning Algorithms Show Structural Connectivity Differences By Amyloid Status.''\\
                             \textbf{Xingjian Zhen}, Rudrasis Chakraborty, Nicholas Vogt, Seong Jae Hwang, Sterling C. Johnson, Barbara B. Bendlin, Vikas Singh. 
                \item \textbf{[\emph{NeurIPS, 2018}]} ``A Statistical Recurrent Model on the Manifold of Symmetric Positive Definite Matrices.''\\
                             Rudrasis Chakraborty, \textbf{Xingjian Zhen}, Chun-Hao Yang, Monami Banerjee, Derek Archer, David Vaillancourt, Vikas Singh, Baba C. Vemuri.
                \end{itemize}

\section{RESEARCH \\ EXPERIENCES}
                \textbf{Flow-based Generative Model for Non-Euclidean Data}\hfill 03/2019-12/2019
                % Advisor: Prof. Vikas Singh, Dept. of Biostatistics and the Dept. of Computer Sciences at the University of Wisconsin-Madison
                \begin{itemize}\itemsep -2.2pt %reduce space between items
                 \item Introduced three invertible layers whose determinant of Jacobian is simple
                 \item Built the two-stream version of GLOW that can transfer information from one manifold to another
                 \item Showed the ability to transfer DTI to corresponding ODF, and vice versa
                 \item Generated/ Mixed texture images based on the covariance matrices
                 \end{itemize}

                \textbf{Point Cloud Completion}\hfill 09/2019-11/2019
                % Advisor: Prof. Vikas Singh, Dept. of Biostatistics and the Dept. of Computer Sciences at the University of Wisconsin-Madison
                \begin{itemize}\itemsep -2.2pt %reduce space between items
                 \item Used encoder-decoder based network to roughly complete the point cloud
                 \item Found nearest neighbor in the training dataset to extract local information
                 \end{itemize}

                \textbf{Dilated CNN in Group Analysis of Alzheimer's Disease} \hfill 08/2018-03/2019
                % Advisor: Prof. Vikas Singh, Dept. of Biostatistics and the Dept. of Computer Sciences at the University of Wisconsin-Madison
                \begin{itemize}\itemsep -2.2pt %reduce space between items
                 % \item Pre-processed the dMRI to extract the information of fiber bundles
                 \item Applied SPD/ODF Manifold into Dilated CNN model to directly extract information from DTI/ ODF
                 \item Speed up the training and testing model with competitive number of parameters with the state of the art 
                 \item With PiB and APOE biomarkers, got statistically significant results on several fiber bundles
                 \end{itemize}

                 \textbf{Statistical Recurrent Model on the Manifold}\hfill 01/2018-05/2018
                % Advisor: Prof. Vikas Singh, Dept. of Biostatistics and the Dept. of Computer Sciences at the University of Wisconsin-Madison
                \begin{itemize}\itemsep -2.2pt %reduce space between items
                 \item Defined the operator in the manifold space
                 \item Applied SPD manifold into statistical recurrent model
                 \item Significantly reduced the number of parameters of the video classification model
                 \item Used this model to achieve the state of art of accuraccy in UCF11 dataset
                 \end{itemize}

                 \textbf{Correlationship for Image-Text Pair in Latent Space}\hfill 09/2017-01/2018
                % Advisor: Prof. Liangrui Peng, Dept. of Electronic Engineering, Tsinghua Univ.
                \begin{itemize}\itemsep -2.2pt %reduce space between items
                 \item Applied the pre-trained CNN as the feature extractor from image side 
                 \item Applied the word2vec method on text as the representive of sentence
                 \item Used t-SNE to minimum the KL divergence between latent space and the Image-Text pair
                 \item Got meaningful result for the local dataset
                 \end{itemize}

\section{INTERN \\ EXPERIENCE}
                \textbf{DAMO Academy, Alibaba}\hfill 05/2019-09/2019
                \begin{itemize}\itemsep -2.2pt %reduce space between items
                 \item Research intern in DAMO Academy Medical AI Algorithm Research, Alibaba
                 \item Developed an automated anatomical labeling of coronary arteries via CPR-GCN
                 \item Used 3D-CNN with BiLSTM to extract the features from the CT images along branches
                 \item Used both image domain and position information with the partial-residual connection to achieve 95.8\% mean recall
                 \end{itemize}
\section{COMPUTER \\ SKILLS} 
Deep learning framework: PyTorch, TensorFlow\\
Languages: Python, C++, Matlab \\
Softwares: Visual Studio, Matlab 


\end{resume}
\end{document}

