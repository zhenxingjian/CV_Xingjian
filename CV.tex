% LaTeX resume using res.cls
\documentclass[margin]{res}
%\usepackage{helvetica} % uses helvetica postscript font (download helvetica.sty)
%\usepackage{newcent}   % uses new century schoolbook postscript font
\setlength{\textwidth}{5.1in} % set width of text portion
% \usepackage [colorlinks,linkcolor=blue,anchorcolor=blue,citecolor=green]{hyperref}
\renewcommand{\baselinestretch}{1.2} 
\usepackage{url}
\usepackage{enumitem}
\usepackage{hyperref}
% \usepackage{setspace}
% \usepackage{parskip} % This package makes the spacing to be wider. -Qi
\begin{document}


% Center the name over the entire width of resume:
 \moveleft.5\hoffset\centerline{\large\bf Xingjian Zhen}
% Draw a horizontal line the whole width of resume:
 \moveleft\hoffset\vbox{\hrule width\resumewidth height 0.1pt}\smallskip
% address begins here
% Again, the address lines must be centered over entire width of resume:
 \moveleft.5\hoffset\centerline{Department of Computer Science, University of Wisconsin-Madison}
 \moveleft.5\hoffset\centerline{5770B Medical Science Center. Madison, WI, 53705}
 \moveleft.5\hoffset\centerline{\url{xzhen3@wisc.edu}}
\begin{resume}
\section{RESEARCH \\ INTERESTS}
                My research interest is about different structured data for medical application in Computer Vision. Since it’s quite mature for Euclidean space machine learning, I would like to explore some different structured data or constrained data for classification. For example, the Symmetric Positive Definite matrix in medical data (DTI) or covariance matrix are the data with constraint. I would like to try some neural networks in Euclidean space and extend them to the manifold data, which is the structured data, to do the classification or regression. I believe this will be useful in diagnosis from a medical perspective or analysis the video/image information.
\section{EDUCATION} 
                {\sl {PhD. Student}}, Department of Computer Science \hfill August 2017 - present\\
                UW-Madison, WI, U.S. \\
                {\sl {B.E.}}, Department of Electronic Engineering \hfill August 2013 - July 2017\\
                Tsinghua University, Beijing, P.R. China. 

\section{Paper}
                \begin{itemize}\itemsep -2.2pt
                \item Rudrasis Chakraborty, \textbf{Xingjian Zhen}, Chun-Hao Yang, Monami Banerjee, Derek Archer, David Vaillancourt, Vikas Singh, Baba C. Vemuri. ``A Statistical Recurrent Model on the Manifold of Symmetric Positive Definite Matrices.''  In \emph{Thirty-second Conference on Neural Information Processing Systems (NeurIPS), 2018}
                \item \textbf{Xingjian Zhen}, Rudrasis Chakraborty, Nicholas Vogt, Seong Jae Hwang, Sterling C. Johnson, Barbara Bendlin, Vikas Singh. ``Group Analysis for PiB Status with Seqential Deep Learning Model on DTI.'' \emph{Alzheimer's Association International Conference (AAIC), 2019}
                \item \textbf{Xingjian Zhen}, Rudrasis Chakraborty, Nicholas Vogt, Barbara Bendlin, Vikas Singh. ``Dilated Convolutional Neural Networks (DCNN) for Sequential Manifold-valued Data in Neuroimaging.'' \emph{IEEE International Conference on Computer Vision (ICCV), 2019}
                \end{itemize}

\section{RESEARCH \\ EXPERIENCES}

                \textbf{Group Analysis for PiB Status with Seqential Deep Learning Model on DTI}\hfill 01/2019-03/2019
                % Advisor: Prof. Vikas Singh, Dept. of Biostatistics and the Dept. of Computer Sciences at the University of Wisconsin-Madison
                \begin{itemize}\itemsep -2.2pt %reduce space between items
                 \item Used Ants as registration tool to warp information from template space into subject space 
                 \item Extracted each voxel along each fiber bundles in DTI space to fit the sequential model
                 \item With PiB status as group, found 2 fiber bundles satisfying significance level
                 \end{itemize}

                \textbf{Dilated CNN in Group Analysis of Alzheimer's Disease} \hfill 08/2018-12/2018
                % Advisor: Prof. Vikas Singh, Dept. of Biostatistics and the Dept. of Computer Sciences at the University of Wisconsin-Madison
                \begin{itemize}\itemsep -2.2pt %reduce space between items
                 \item Pre-processed the dMRI to extract the information of fiber bundles
                 \item Applied SPD/ODF Manifold into Dilated CNN model to directly extract information from DTI/ODF
                 \item Speed up the training and testing model with competitive number of parameters with the state of the art 
                 \item With CSF and APOE biomarkers, got statistically significant results on several fiber bundles
                 \end{itemize}

                 \textbf{Statistical Recurrent Model on the Manifold}\hfill 01/2018-05/2018
                % Advisor: Prof. Vikas Singh, Dept. of Biostatistics and the Dept. of Computer Sciences at the University of Wisconsin-Madison
                \begin{itemize}\itemsep -2.2pt %reduce space between items
                 \item Defined the operator in the manifold space
                 \item Applied SPD manifold into statistical recurrent model
                 \item Significantly reduced the number of parameters of the video classification model
                 \item Used this model to achieve the state of art of accuraccy in UCF11 dataset
                 \end{itemize}

                 \textbf{Correlationship for Image-Text Pair in Latent Space}\hfill 09/2017-01/2018
                % Advisor: Prof. Liangrui Peng, Dept. of Electronic Engineering, Tsinghua Univ.
                \begin{itemize}\itemsep -2.2pt %reduce space between items
                 \item Applied the pre-trained CNN as the feature extractor from image side 
                 \item Applied the word2vec method on text as the representive of sentence
                 \item Used t-SNE to minimum the KL divergence between latent space and the Image-Text pair
                 \item Got meaningful result for the local dataset
                 \end{itemize}

                \textbf{Form Line Detection in the Picture}\hfill 12/2016-06/2017
                % Advisor: Prof. Liangrui Peng, Dept. of Electronic Engineering, Tsinghua Univ.
                \begin{itemize}\itemsep -2.2pt %reduce space between items
                 \item Developed a system that detects and recognizes the form lines in pictures
                 \item Used the bidirectional RNN method to achieve the state of the art, with MXNet as the core of the deep-learning system
                 \item Tested in multiple databases such as the NIST Special Database 2 and got a high accuracy rate
                 \end{itemize}
                
                \textbf{Emotion Detection in Voice Recordings}\hfill 07/2016-08/2016 
                % Advisor: Prof. Christian Poellabauer, Dept. of Computer Science and Engineering, Univ. of Notre Dame
                \begin{itemize}\itemsep -2.2pt %reduce space between items
                 \item Used different classifiers to predict the emotion of a speaker in a voice memo 
                 \item Extracted voice features like MFCC, LPCC, Zero Cross Rate, etc., from FAU Aibo Emotion Corpus dataset
                 \item Trained both an SVM and a neural network using the features, tested them, and compared their accuracy with the state-of-the-art
                 % \item Both the SVM and the neural network methods reached the state-of-the-art performance in the Interspeech 2009 Emotion Challenge
                 \end{itemize}

\section{INTERN \\ EXPERIENCES}
                \textbf{DAMO Academy, Alibaba}\hfill 05/2019-09/2019
                \begin{itemize}\itemsep -2.2pt %reduce space between items
                 \item Research intern in DAMO Academy Medical AI Algorithm Research, Alibaba
                 \item Developed an automated anatomical labeling of coronary arteries via graph neural networks
                 \item Used 3D-CNN with LSTM model to extract the features from the CT images along vessels
                 \item Used both image domain and position information to classify the coronary arteries and achieved 93\% accuracy on the test set
                 \end{itemize}
\section{COMPUTER \\ SKILLS} 
Languages: Python, C++, Matlab \\
Softwares: Visual Studio, Matlab \\
Deep learning framework: PyTorch, TensorFlow



\end{resume}
\end{document}

