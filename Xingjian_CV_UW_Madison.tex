% LaTeX resume using res.cls
\documentclass[margin]{res}
\usepackage{CJKutf8}
\makeatletter
\def\@classoptionslist{<class options except `margin` OR empty>}
\makeatother

\usepackage{geometry}
\geometry{left=1cm,right=1cm,top=2cm,bottom=2cm}

%\usepackage{helvetica} % uses helvetica postscript font (download helvetica.sty)
%\usepackage{newcent}   % uses new century schoolbook postscript font
\setlength{\textwidth}{6.3in} % set width of text portion
% \usepackage [colorlinks,linkcolor=blue,anchorcolor=blue,citecolor=green]{hyperref}
\renewcommand{\baselinestretch}{1.2} 
\usepackage{url}
\usepackage{enumitem}
\usepackage{varioref}
\usepackage{hyperref}
\usepackage{cleveref}
\usepackage{xcolor}
\usepackage{array}
% \usepackage{setspace}
% \usepackage{parskip} % This package makes the spacing to be wider. -Qi
\usepackage{fontawesome}
\usepackage{cleveref}

\renewcommand\labelenumi{[\theenumi]}

\begin{document}

% Center the name over the entire width of resume:
% \moveleft.45\hoffset\centerline{\large\bf Xingjian Zhen (\begin{CJK*}{UTF8}{gbsn}甄\ 行践\end{CJK*})}
\moveleft.45\hoffset\centerline{\large\bf Xingjian ZHEN}
% Draw a horizontal line the whole width of resume:
 \moveleft\hoffset\vbox{\hrule width\resumewidth height 1pt}\smallskip
% address begins here
% Again, the address lines must be centered over entire width of resume:
 % \moveleft.45\hoffset\centerline{Department of Computer Science, University of Wisconsin-Madison}
 % \moveleft.45\hoffset\centerline{5770B Medical Science Center. Madison, WI, 53705}
 % \moveleft.45\hoffset\centerline{\href{mailto:xzhen3@wisc.edu}{xzhen3@wisc.edu}}
 % \moveleft.45\hoffset\centerline{\href{https://zhenxingjian.github.io/homepage/}{\textcolor{blue}{[Homepage]\footnote{\url{https://zhenxingjian.github.io/homepage/}}}} \href{https://github.com/zhenxingjian}{\textcolor{blue}{[Github]\footnote{\url{https://github.com/zhenxingjian}}}}}

%\moveleft1.06\hoffset\vbox{
%\begin{tabular}{p{0.3\resumewidth}>{\raggedleft\arraybackslash}p{0.677\resumewidth}}
%    5770B Medical Science Center & \href{mailto:xzhen3@wisc.edu}{\faEnvelope\ xzhen3@wisc.edu}\\
%    1300 University Ave. & \faPhone\ (xxx)-xxx-xxxx\\
%    Madison, WI, USA 53706-1510 & \href{https://zhenxingjian.github.io/homepage}{\faHome\ https://zhenxingjian.github.io/homepage}\\
%    & \href{https://github.com/zhenxingjian}{\faGithub \ https://github.com/zhenxingjian}
%\end{tabular}
%}

\moveleft1.06\hoffset\vbox{
\begin{tabular}{p{0.677\resumewidth}>{\raggedleft\arraybackslash}p{0.3\resumewidth}}
    \href{mailto:xzhen3@wisc.edu}{\faEnvelope\ xzhen3@wisc.edu} & 5770B Medical Science Center \\
    \faPhone\ (xxx)-xxx-xxxx & 1300 University Ave. \\
    \href{https://zhenxingjian.github.io/homepage}{\faHome\ https://zhenxingjian.github.io/homepage} & Madison, WI, USA  \\
    \href{https://github.com/zhenxingjian}{\faGithub \ https://github.com/zhenxingjian} & 53706-1510
\end{tabular}
}


\begin{resume}
\vspace{-2em}
\section{RESEARCH \\ INTEREST}
                \textbf{Computer Vision, Deep Learning, Statistics, Multi-modality, and Medical Imaging}\\
                My research interest is two-fold. On the algorithm side, I'm interested in structured data, such as covariance matrices of features and graphs. Those features are constrained by their own structure, i.e., Symmetric Positive Definite, and break most basic operations in Euclidean space. But the key benefit is that structured data are more robust to noise and contain more high-level information. On the application side, I would like to explore such data structures in medical imaging and multi-modality (such as image-text). In the medical imaging setting, structure information is sensitive and enables us to detect micro changes in the early stage of the disease. More importantly, since there naturally exists structures among modalities, we can use the structures to improve the multi-modality performance (e.g. VQA).                 
%                My research interest is about structured data or high-order features in Computer Vision. Since it's quite mature for Euclidean space deep learning, I would like to explore some different structured data or constrained data. The covariance matrices of the feature vectors can be viewed as the second-order features. Those high-order features are constrained by their own structure, i.e., Symmetric Positive Definite. Thus this will break the basic rules in Euclidean space. But those matrices are more robust to noise and contain more high-level information. I want to try some neural networks in Euclidean space, and extend them to those structured data. 
                
\vspace{-1em}
\section{EDUCATION} 
                {\bf University of Wisconsin-Madison}, Madison, WI, U.S. \hfill 08/2017-11/2022\\
                {\sl {Ph.D. Candidate}}, Department of Computer Science \hfill GPA: 3.85/4.0\\
                {Advisor:} Prof. Vikas Singh\\
                {Doctoral Minor:} Mathematics\\
                {\bf Tsinghua University}, Beijing, P.R. China. \hfill 08/2013-06/2017\\
                {\sl {B.E.}}, Department of Electronic Engineering \hfill GPA: 90.9/100

\vspace{-1em}                
\section{PUBLICATIONS \\ \small{ 2 {\textcolor{red}{Oral}}\\ 5 Poster\\ 2 In Submission \\ 2 Abstract (Omit) \\ 58 Citations} }
                \begin{enumerate}[noitemsep,wide=0pt,leftmargin=\dimexpr\labelwidth + 2\labelsep\relax]\itemsep -0.0pt
                \item \textbf{[\emph{In Submission}]} ``Variational Sampling of Temporal Trajectories.''\\
                				Nazarovs J., Huang Z., \textbf{Zhen X.}, Pal S., Chakraborty R., 
                            and Singh V.
                \item \textbf{[\emph{In Submission}]} ``Frank-Wolfe based Anytime Neural Networks.''\\
                				Meng Z., \textbf{Zhen X.}, Ravi S., and Singh V.           
                \item \textbf{[\emph{ECCV, 2022}]} ``On the Versatile Uses of Partial Distance Correlation in Deep Learning.''\\
                            \textbf{Zhen X.}, Meng Z., Chakraborty R., and Singh V.
                \item \textbf{[\emph{NAACL-MAI-Workshop, 2021}]} ``A First Look: Towards Explainable TextVQA Models via Visual and Textual Explanations.''\\
                            \textbf{Zhen X.}$^*$, Rao V.N.$^*$, Hovsepian K., and Shen M.
%                \item \textbf{[\emph{AAIC, 2021}]} ``Altered Structural Connectivity Detected with Dilated Convolutional Neural Network Analysis in the DIAN study and the Wisconsin Registry for Alzheimer's Prevention.''\\
%                            \textbf{Zhen X.}, Chakraborty R., Vogt N., Wang R., Yang K.L., Adluru N., Gordon B., Benzinger T., Mckay N., Betthauser T., Johnson S.C., Singh V., and Bendlin B.B.
                \item \textbf{[\emph{CVPR \textcolor{red}{(Oral)}, 2021}]} ``Simpler Certified Radius Maximization by Propagating Covariances.''\\
                            \textbf{Zhen X.}, Chakraborty R., and Singh V. 
                \item \textbf{[\emph{AAAI, 2021}]} ``Flow-based Generative Models for Learning Manifold to Manifold Mappings.''\\
                            \textbf{Zhen X.}, Chakraborty R., Yang L., and Singh V. 
                \item \textbf{[\emph{CVPR \textcolor{red}{(Oral)}, 2020}]} ``CPR-GCN: Conditional Partial-Residual Graph Convolutional Network in Automated Anatomical Labeling of Coronary Arteries.''\\
                            \textbf{Zhen X.}$^*$, Yang H.$^*$, Chi Y., Zhang L., and Hua X.S.   
                \item \textbf{[\emph{ICCV, 2019}]} ``Dilated Convolutional Neural Networks for Sequential Manifold-valued Data.'' \\
                             \textbf{Zhen X.}$^*$, Chakraborty R.$^*$, Vogt N., Bendlin B.B., and Singh V. 
%                \item \textbf{[\emph{AAIC, 2019}]} ``Sequential Deep Learning Algorithms Show Structural Connectivity Differences By Amyloid Status.''\\
%                             \textbf{Zhen X.}, Chakraborty C., Vogt N., Hwang S.J., Johnson S.C., Bendlin B.B., and Singh V. 
                \item \textbf{[\emph{NeurIPS, 2018}]} ``A Statistical Recurrent Model on the Manifold of Symmetric Positive Definite Matrices.''\\
                             Chakraborty R., \textbf{Zhen X.}$^*$, Yang C.H.$^*$, Banerjee M., Archer D., Vaillancourt D., Singh V., and Vemuri B.C.
%                \item 
                \end{enumerate}

%\vspace{-2em}                
%\section{UNDER\\ REVIEWING}
%				\begin{enumerate}[noitemsep,wide=0pt,leftmargin=\dimexpr\labelwidth + 2\labelsep\relax]\itemsep -0.0pt
%				\setcounter{enumi}{7}
%%                \item \textbf{[\emph{In submission}]} ``Can Kernel Transfer Operators Help Flow based Generative Models?''\\
%%                            Zhichun Huang, Rudrasis Chakraborty, \textbf{Xingjian Zhen}, Vikas Singh.
%                \end{enumerate}
\vspace{-1em}
\section{COMPUTER \\ SKILLS} 
                Deep learning framework: PyTorch, TensorFlow\\
                Languages: Python, \LaTeX
\vspace{-1em}
\section{INTERN \\ EXPERIENCES}
				\textbf{Amazon}, Pasadena, CA, U.S. \hfill 05/2021-08/2021\\
                {\it Applied Scientist Intern}, AWS Textract\\
                \textbf{Mentor:} Vijay Mahadevan, Zhuowen Tu, and Qi Dong
%                \textbf{Title:} Applied Scientist Intern
                \begin{itemize}\itemsep -2.2pt %reduce space between items
                \item[-] Introduced transformer-based agent method that takes pair-wise input to do the co-detection
                \item[-] In the small in-house dataset, our method improved F1 score 6\% from the current SOTA, while in the entire dataset, it is 0.5\% better
                \item[-] In the MS-COCO dataset, our method beats Deformable DETR on small dataset by 4\% 
                 \end{itemize}
                
                \vspace{-1em}
                \textbf{Amazon}, Seattle, WA, U.S. \hfill 05/2020-09/2020\\
                {\it Applied Scientist Intern}, Product Assurance Risk Security ML Group\\
                \textbf{Mentor:} Karen Hovsepian, and Mingwei Shen
%                \textbf{Title:} Applied Scientist Intern
                \begin{itemize}\itemsep -2.2pt %reduce space between items
                \item[-] Built an end-to-end three-mode model MTXNet to generate answer, textual explanation, and visual saliency explanation, with graph neural network and the noise augmentation based on M4C
                \item[-] Got $1\%$ better accuracy, $7\%$ better textual explanation CIDEr, $2\%$ better visual explanation IoU
                 \item[-] Collected a novel TextVQA-X dataset from public available TextVQA with further explanation
                 \item[-] Published paper [4] in NAACL-MAI-Workshop 2021, together with oral presentation internally
                 \end{itemize}
                
                \vspace{-1em}  
                
                \textbf{DAMO Academy, Alibaba}, Beijing, P.R. China. \hfill 05/2019-09/2019\\
                {\it Research Intern}, Medical AI Algorithm Research Group\\
                \textbf{Mentor:} Ying Chi
%                \textbf{Title:} Research Intern
                \begin{itemize}\itemsep -2.2pt %reduce space between items
                 % \item[-] Research intern in DAMO Academy Medical AI Algorithm Research, Alibaba
                 \item[-] Developed automated anatomical labeling of coronary arteries via CPR-GCN
                 \item[-] Used 3D CNN with BiLSTM to extract the features from the CT images along branches
                 \item[-] Combined both image domain and position information with the partial-residual connection over GCN to achieve $95.8\%$ mean recall, $9\%$ improvement from baseline
                 \item[-] Published paper [7] with oral presentation in CVPR 2020 ($5.7\%$ acceptance rate)
                 \end{itemize}

\vspace{-1em}
\section{RESEARCH \\ EXPERIENCES}
				\textbf{Partial Distance Correlation (PDC) in Deep Learning and the Benefit}\hfill 12/2020-11/2021
                % Advisor: Prof. Vikas Singh, Dept. of Biostatistics and the Dept. of Computer Sciences at the University of Wisconsin-Madison
                \begin{itemize}\itemsep -2.2pt %reduce space between items
                 \item[-] Introduced DC into robustness so that the transferred attack accuracy under PGD drops 9\%
                 \item[-] Enabled training a network conditioned on another by Partial DC, and analysis ViT over ResNet
                 \item[-] Used DC to disentangle the latent representation, and generated SOTA manipulate image on FFHD
                 \item[-] Published paper [3] in ECCV 2022 ($28\%$ acceptance rate)
                 \end{itemize}

                 \vspace{-1em} 
                \textbf{Certified Robustness Training via Propagating Gaussian Distribution}\hfill 01/2020-11/2020
                % Advisor: Prof. Vikas Singh, Dept. of Biostatistics and the Dept. of Computer Sciences at the University of Wisconsin-Madison
                \begin{itemize}\itemsep -2.2pt %reduce space between items
                 \item[-] Applied the certifiable randomized smoothing robustness without sampling with $2\times$ faster
                 \item[-] Proposed a tight estimation of the channel-wise Gaussian distribution to reduce computational cost
                 \item[-] Achieved better certified accuracy and $5\%$ larger certified radius on ImageNet and Places365
                 \item[-] Published paper [5] with oral presentation in CVPR 2021 ($4.6\%$ acceptance rate)
                 \end{itemize}

                 \vspace{-1em}  

                \textbf{Flow-based Generative Model for Non-Euclidean Data}\hfill 03/2019-12/2019
                % Advisor: Prof. Vikas Singh, Dept. of Biostatistics and the Dept. of Computer Sciences at the University of Wisconsin-Madison
                \begin{itemize}\itemsep -2.2pt %reduce space between items
                 \item[-] Introduced three invertible layers on manifold-valued data whose determinant of Jacobian is simple
                 \item[-] Built the two-stream GLOW that can transfer information from one manifold to another
                 \item[-] Transferred DTI to corresponding ODF with a small reconstruction error and maintaining verifiable group difference with $p-$value $<0.001$
%                 In reality, this can save scanning time from $35$ mins to $7-9$ mins
				\item[-] Published paper [6] in AAAI 2021 ($21\%$ acceptance rate)
                 \end{itemize}

                 \vspace{-1em}  

%                \textbf{Point Cloud Completion}\hfill 09/2019-11/2019
%                % Advisor: Prof. Vikas Singh, Dept. of Biostatistics and the Dept. of Computer Sciences at the University of Wisconsin-Madison
%                \begin{itemize}\itemsep -2.2pt %reduce space between items
%                 \item[-] Used the encoder-decoder based network to roughly complete the point cloud
%                 \item[-] Utilized the nearest neighbor in the training dataset to extract local information
%                 \end{itemize}

                \textbf{Manifold Dilated CNN in Group Analysis of Alzheimer's Disease} \hfill 08/2018-02/2021
                % Advisor: Prof. Vikas Singh, Dept. of Biostatistics and the Dept. of Computer Sciences at the University of Wisconsin-Madison
                \begin{itemize}\itemsep -2.2pt %reduce space between items
                 \item[-] Used TractSeg and Tractometry to compute the average representation along 50 fiber bundles
                 \item[-] Introduced SPD/ $S^n$ manifold into the Dilated CNN model to extract information from DTI/ ODF
                 \item[-] Sped up the training and testing $5\times$ with a competitive number of parameters with SoTA
                 \item[-] Got statistically significant difference on 14 and 16 (out of 50) fiber bundles, by PiB-PET and Gene mutation carriers, on DIAN and WRAP dataset, with total 9 fiber bundles in common
                 \item[-] Published paper [8] in ICCV 2019 ($25\%$ acceptance rate), and also extended to 2 AAIC abstracts
                 \end{itemize}

%                 \vspace{-1em}  
%
%                 \textbf{Statistical Recurrent Model on the SPD Manifold} \hfill 01/2018-05/2018
%                % Advisor: Prof. Vikas Singh, Dept. of Biostatistics and the Dept. of Computer Sciences at the University of Wisconsin-Madison
%                \begin{itemize}\itemsep -2.2pt %reduce space between items
%                 \item[-] Defined the ``$+/\ \times$'' operators in the manifold space using group operators and wFM
%                 \item[-] Reduced the number of parameters of the video classification model $100\times$
%                 \item[-] Achieved SoTA accuracy, $78\%$, on UCF11 dataset
%                 \item[-] Published paper [8] in NeurIPS 2018 ($21\%$ acceptance rate)
%                 \end{itemize}

%                 \vspace{-1em}  

%                 \textbf{Correlations for Image-Text Pair in Latent Space} \hfill 09/2017-01/2018
%                \begin{itemize}\itemsep -2.2pt %reduce space between items
%                 \item[-] Applied the pre-trained CNN as the feature extractor from the image side 
%                 \item[-] Applied the word2vec method on the text as the representative of text
%                 \item[-] Used t-SNE to minimize the KL divergence between latent space and the Image-Text pair
%                 \item[-] Got meaningful results on the local dataset gathered from Reddit
%                 \end{itemize}

%                 \textbf{Form Line Detection in the Picture} \hfill 12/2016-06/2017
%                % Advisor: Prof. Liangrui Peng, Dept. of Electronic Engineering, Tsinghua Univ.
%                \begin{itemize}\itemsep -2.2pt %reduce space between items
%                 \item[-] Developed a system that detects and recognizes the form lines in pictures
%                 \item[-] Used the bidirectional RNN method to achieve the state of the art, with MXNet as the core of the deep-learning system 
%                 \item[-] Tested in multiple dataset such as the NIST Special Database 2 and got $99.0\%$ accuracy rate
%                 \end{itemize}

%
%\vspace{-1em}
%\section{REVIEWER \\ SERVICES}
%                2020: ECCV, MICCAI, NeurIPS,\\ 2021: AAAI, ICLR, CVPR, IJCAI, ICML, MICCAI, ICCV, NeurIPS

\end{resume}
\end{document}

