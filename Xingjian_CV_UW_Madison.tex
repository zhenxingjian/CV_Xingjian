% LaTeX resume using res.cls
\documentclass[margin]{res}
\makeatletter
\def\@classoptionslist{<class options except `margin` OR empty>}
\makeatother

\usepackage{geometry}
\geometry{left=1cm,right=1cm,top=2cm,bottom=2cm}

%\usepackage{helvetica} % uses helvetica postscript font (download helvetica.sty)
%\usepackage{newcent}   % uses new century schoolbook postscript font
\setlength{\textwidth}{6.3in} % set width of text portion
% \usepackage [colorlinks,linkcolor=blue,anchorcolor=blue,citecolor=green]{hyperref}
\renewcommand{\baselinestretch}{1.2} 
\usepackage{url}
\usepackage{enumitem}
\usepackage{hyperref}
\usepackage{xcolor}
\usepackage{array}
% \usepackage{setspace}
% \usepackage{parskip} % This package makes the spacing to be wider. -Qi
\begin{document}


% Center the name over the entire width of resume:
 \moveleft.45\hoffset\centerline{\large\bf Xingjian Zhen}
% Draw a horizontal line the whole width of resume:
 \moveleft\hoffset\vbox{\hrule width\resumewidth height 1pt}\smallskip
% address begins here
% Again, the address lines must be centered over entire width of resume:
 % \moveleft.45\hoffset\centerline{Department of Computer Science, University of Wisconsin-Madison}
 % \moveleft.45\hoffset\centerline{5770B Medical Science Center. Madison, WI, 53705}
 % \moveleft.45\hoffset\centerline{\href{mailto:xzhen3@wisc.edu}{xzhen3@wisc.edu}}
 % \moveleft.45\hoffset\centerline{\href{https://zhenxingjian.github.io/homepage/}{\textcolor{blue}{[Homepage]\footnote{\url{https://zhenxingjian.github.io/homepage/}}}} \href{https://github.com/zhenxingjian}{\textcolor{blue}{[Github]\footnote{\url{https://github.com/zhenxingjian}}}}}

\moveleft1.06\hoffset\vbox{
\begin{tabular}{p{0.5\resumewidth}>{\raggedleft\arraybackslash}p{0.477\resumewidth}}
    5770B Medical Science Center & \href{mailto:xzhen3@wisc.edu}{xzhen3@wisc.edu}\\
    1300 University Ave. & \\
    Madison, WI, USA 53706-1510 & \href{https://zhenxingjian.github.io/homepage/}{\textcolor{blue}{[Homepage]\footnotemark[1]}} \href{https://github.com/zhenxingjian}{\textcolor{blue}{[Github]\footnotemark[2]}}
\end{tabular}
}

\footnotetext[1]{\url{https://zhenxingjian.github.io/homepage/}}
\footnotetext[2]{\url{https://github.com/zhenxingjian}}

\begin{resume}
\vspace{-2em}
\section{RESEARCH \\ INTEREST}
                \textbf{Computer Vision, Deep Learning, Statistics, and Medical Imaging}\\
                My research interest is about structured data or high-order features in Computer Vision. Since it's quite mature for Euclidean space deep learning, I would like to explore some different structured data or constrained data. The covariance matrices of the feature vectors can be viewed as the second-order features. Those high-order features are constrained by their own structure, i.e., Symmetric Positive Definite. Thus this will break the basic rules in Euclidean space. But those matrices are more robust to noise and contain more high-level information. I want to try some neural networks in Euclidean space, and extend them to those structured data.
                
\vspace{-1em}
\section{EDUCATION} 
                {\bf University of Wisconsin-Madison}, WI, U.S. \hfill 2017 - 2022(Expected)\\
                {\sl {Ph.D. Student}}, Department of Computer Science \hfill GPA 3.83/ 4.0\\
                {\bf Supervisor:} Prof. Vikas Singh\\
                {\bf Tsinghua University}, Beijing, P.R. China. \hfill 2013 - 2017\\
                {\sl {B.E.}}, Department of Electronic Engineering \hfill GPA 90.9/100

\vspace{-1em}                
\section{PUBLICATIONS}
                \begin{itemize}[noitemsep,wide=0pt,leftmargin=\dimexpr\labelwidth + 2\labelsep\relax]\itemsep -0.0pt
                \item \textbf{[\emph{In submission}]} ``A First Look: Towards Explainable TextVQA Models via Visual and Textual Explanations.''\\
                            \textbf{Xingjian Zhen}, Varun Nagaraj Rao, Karen Hovsepian, Mingwei Shen.
                \item \textbf{[\emph{In submission}]} ``Can Kernel Transfer Operators Help Flow based Generative Models?''\\
                            Zhichun Huang, Rudrasis Chakraborty, \textbf{Xingjian Zhen}, Vikas Singh.
                \item \textbf{[\emph{In submission}]} ``Simpler Certified Radius Maximization by Propagating Covariances.''\\
                            \textbf{Xingjian Zhen}, Rudrasis Chakraborty, Vikas Singh.
                \item \textbf{[\emph{AAAI, 2021}]} ``Flow-based Generative Models for Learning Manifold to Manifold Mappings.''\\
                            \textbf{Xingjian Zhen}, Rudrasis Chakraborty, Liu Yang, Vikas Singh.   
                \item \textbf{[\emph{CVPR \textcolor{red}{(Oral)}, 2020}]} ``CPR-GCN: Conditional Partial-Residual Graph Convolutional Network in Automated Anatomical Labeling of Coronary Arteries.''\\
                            \textbf{Xingjian Zhen}, Han Yang, Ying Chi, Lei Zhang, Xian-Sheng Hua.   
                \item \textbf{[\emph{ICCV, 2019}]} ``Dilated Convolutional Neural Networks for Sequential Manifold-valued Data.'' \\
                             \textbf{Xingjian Zhen}, Rudrasis Chakraborty, Nicholas Vogt, Barbara B. Bendlin, Vikas Singh. 
                \item \textbf{[\emph{AAIC, 2019}]} ``Sequential Deep Learning Algorithms Show Structural Connectivity Differences By Amyloid Status.''\\
                             \textbf{Xingjian Zhen}, Rudrasis Chakraborty, Nicholas Vogt, Seong Jae Hwang, Sterling C. Johnson, Barbara B. Bendlin, Vikas Singh. 
                \item \textbf{[\emph{NeurIPS, 2018}]} ``A Statistical Recurrent Model on the Manifold of Symmetric Positive Definite Matrices.''\\
                             Rudrasis Chakraborty, \textbf{Xingjian Zhen}, Chun-Hao Yang, Monami Banerjee, Derek Archer, David Vaillancourt, Vikas Singh, Baba C. Vemuri.
                \end{itemize}

\vspace{-1em}
\section{INTERN \\ EXPERIENCE}
                \textbf{Amazon}\hfill 05/2020-09/2019\\
                \textbf{Mentor:} Karen Hovsepian, and Mingwei Shen\\
                \textbf{Title:} Applied Scientist Intern
                \begin{itemize}\itemsep -2.2pt %reduce space between items
                \item[-] Built an end-to-end three-modes model to generate answer, textual explanation, and visual saliency explanation with graph neural network and the noise augmentation to improve the robustness
                \item[-] Got $1\%$ better answer accuracy, $7\%$ better textual explanation CIDEr, $2\%$ better visual explanation IoU
                 \item[-] Collected a novel TextVQA-X dataset from public available TextVQA with extra textual and visual explanation
                 \end{itemize}
                 
                \textbf{DAMO Academy, Alibaba}\hfill 05/2019-09/2019\\
                \textbf{Mentor:} Ying Chi\\
                \textbf{Title:} Research Intern
                \begin{itemize}\itemsep -2.2pt %reduce space between items
                 % \item[-] Research intern in DAMO Academy Medical AI Algorithm Research, Alibaba
                 \item[-] Developed automated anatomical labeling of coronary arteries via CPR-GCN
                 \item[-] Used 3D CNN with BiLSTM to extract the features from the CT images along branches
                 \item[-] Combined both image domain and position information with the partial-residual connection to achieve 95.8\% mean recall, $9\%$ improvement from baseline
                 \end{itemize}

\vspace{-1em}
\section{RESEARCH \\ EXPERIENCES}
                \textbf{Certified Robustness Training via Gaussian Distribution}\hfill 01/2020-current
                % Advisor: Prof. Vikas Singh, Dept. of Biostatistics and the Dept. of Computer Sciences at the University of Wisconsin-Madison
                \begin{itemize}\itemsep -2.2pt %reduce space between items
                 \item[-] Applied the certifiable randomized smoothing robustness without sampling with $2\times$ to $5\times$ faster
                 \item[-] Proposed a tight estimation of the channel-wise Gaussian distribution to reduce the cost from exponential to linear
                 \item[-] Achieved $5\%$ better certified accuracy as well as a larger certified radius on ImageNet and Places365
                 \end{itemize}

                \textbf{Flow-based Generative Model for Non-Euclidean Data}\hfill 03/2019-12/2019
                % Advisor: Prof. Vikas Singh, Dept. of Biostatistics and the Dept. of Computer Sciences at the University of Wisconsin-Madison
                \begin{itemize}\itemsep -2.2pt %reduce space between items
                 \item[-] Introduced three invertible layers on manifold-valued data whose determinant of Jacobian is simple
                 \item[-] Built the two-stream GLOW that can transfer information from one manifold to another
                 \item[-] Transferred DTI to corresponding ODF with a small reconstruction error and maintaining verifiable group differences with $p-$value $<0.001$\\ 
                 In reality, this can save scanning time from $35$ mins to $7-9$ mins
                 \end{itemize}

%                \textbf{Point Cloud Completion}\hfill 09/2019-11/2019
%                % Advisor: Prof. Vikas Singh, Dept. of Biostatistics and the Dept. of Computer Sciences at the University of Wisconsin-Madison
%                \begin{itemize}\itemsep -2.2pt %reduce space between items
%                 \item[-] Used the encoder-decoder based network to roughly complete the point cloud
%                 \item[-] Utilized the nearest neighbor in the training dataset to extract local information
%                 \end{itemize}

                \textbf{Dilated CNN in Group Analysis of Alzheimer's Disease} \hfill 08/2018-03/2019
                % Advisor: Prof. Vikas Singh, Dept. of Biostatistics and the Dept. of Computer Sciences at the University of Wisconsin-Madison
                \begin{itemize}\itemsep -2.2pt %reduce space between items
                 \item[-] Pre-processed the dMRI to extract the centerline/ average values along fiber bundles
                 \item[-] Applied SPD/ $S^n$ manifold into the Dilated CNN model to extract information from DTI/ ODF
                 \item[-] Sped up the training and testing $5\times$ with a competitive number of parameters with SoTA
                 \item[-] Got statistically significant differences on 5 (out of 18) fiber bundles, by PiB and APOE biomarkers
                 \end{itemize}

                 \textbf{Statistical Recurrent Model on the Manifold} \hfill 01/2018-05/2018
                % Advisor: Prof. Vikas Singh, Dept. of Biostatistics and the Dept. of Computer Sciences at the University of Wisconsin-Madison
                \begin{itemize}\itemsep -2.2pt %reduce space between items
                 \item[-] Defined the ``$+/\ \times$'' operators in the manifold space
                 \item[-] Modified the statistical recurrent model on the SPD manifold
                 \item[-] Reduced the number of parameters of the video classification model $100\times$
                 \item[-] Achieved the state of the art of accuracy in UCF11 dataset
                 \end{itemize}

                 \textbf{Correlations for Image-Text Pair in Latent Space} \hfill 09/2017-01/2018
                \begin{itemize}\itemsep -2.2pt %reduce space between items
                 \item[-] Applied the pre-trained CNN as the feature extractor from the image side 
                 \item[-] Applied the word2vec method on the text as the representative of text
                 \item[-] Used t-SNE to minimize the KL divergence between latent space and the Image-Text pair
                 \item[-] Got meaningful results on the local dataset gathered from Reddit
                 \end{itemize}

%                 \textbf{Form Line Detection in the Picture} \hfill 12/2016-06/2017
%                % Advisor: Prof. Liangrui Peng, Dept. of Electronic Engineering, Tsinghua Univ.
%                \begin{itemize}\itemsep -2.2pt %reduce space between items
%                 \item[-] Developed a system that detects and recognizes the form lines in pictures
%                 \item[-] Used the bidirectional RNN method to achieve the state of the art, with MXNet as the core of the deep-learning system 
%                 \item[-] Tested in multiple dataset such as the NIST Special Database 2 and got $99.0\%$ accuracy rate
%                 \end{itemize}

\vspace{-1em}
\section{COMPUTER \\ SKILLS} 
                Deep learning framework: PyTorch, TensorFlow, MXNet\\
                Languages: Python, \LaTeX, Matlab, C++

\vspace{-1em}
\section{REVIEWER \\ SERVICES}
                ECCV 2020, MICCAI 2020, NeurIPS 2020, AAAI 2021, ICLR 2021

\end{resume}
\end{document}

