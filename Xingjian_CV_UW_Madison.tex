% LaTeX resume using res.cls
\documentclass[margin]{res}
\usepackage{CJKutf8}
\makeatletter
\def\@classoptionslist{<class options except `margin` OR empty>}
\makeatother

\usepackage{geometry}
\geometry{left=1cm,right=1cm,top=2cm,bottom=2cm}

%\usepackage{helvetica} % uses helvetica postscript font (download helvetica.sty)
%\usepackage{newcent}   % uses new century schoolbook postscript font
\setlength{\textwidth}{6.3in} % set width of text portion
% \usepackage [colorlinks,linkcolor=blue,anchorcolor=blue,citecolor=green]{hyperref}
\renewcommand{\baselinestretch}{1.2} 
\usepackage{url}
\usepackage{enumitem}
\usepackage{varioref}
\usepackage{hyperref}
\usepackage{cleveref}
\usepackage{xcolor}
\usepackage{array}
% \usepackage{setspace}
% \usepackage{parskip} % This package makes the spacing to be wider. -Qi
\usepackage{fontawesome}
\usepackage{cleveref}

\renewcommand\labelenumi{[\theenumi]}

\begin{document}

% Center the name over the entire width of resume:
% \moveleft.45\hoffset\centerline{\large\bf Xingjian Zhen (\begin{CJK*}{UTF8}{gbsn}甄\ 行践\end{CJK*})}
\moveleft.45\hoffset\centerline{\large\bf Xingjian ZHEN}
% Draw a horizontal line the whole width of resume:
 \moveleft\hoffset\vbox{\hrule width\resumewidth height 1pt}\smallskip
% address begins here
% Again, the address lines must be centered over entire width of resume:
 % \moveleft.45\hoffset\centerline{Department of Computer Science, University of Wisconsin-Madison}
 % \moveleft.45\hoffset\centerline{5770B Medical Science Center. Madison, WI, 53705}
 % \moveleft.45\hoffset\centerline{\href{mailto:xzhen3@wisc.edu}{xzhen3@wisc.edu}}
 % \moveleft.45\hoffset\centerline{\href{https://zhenxingjian.github.io/homepage/}{\textcolor{blue}{[Homepage]\footnote{\url{https://zhenxingjian.github.io/homepage/}}}} \href{https://github.com/zhenxingjian}{\textcolor{blue}{[Github]\footnote{\url{https://github.com/zhenxingjian}}}}}

%\moveleft1.06\hoffset\vbox{
%\begin{tabular}{p{0.3\resumewidth}>{\raggedleft\arraybackslash}p{0.677\resumewidth}}
%    5770B Medical Science Center & \href{mailto:xzhen3@wisc.edu}{\faEnvelope\ xzhen3@wisc.edu}\\
%    1300 University Ave. & \faPhone\ (xxx)-xxx-xxxx\\
%    Madison, WI, USA 53706-1510 & \href{https://zhenxingjian.github.io/homepage}{\faHome\ https://zhenxingjian.github.io/homepage}\\
%    & \href{https://github.com/zhenxingjian}{\faGithub \ https://github.com/zhenxingjian}
%\end{tabular}
%}

\moveleft1.06\hoffset\vbox{
\begin{tabular}{p{0.677\resumewidth}>{\raggedleft\arraybackslash}p{0.3\resumewidth}}
    \href{mailto:zhenxingjian1995@gmail.com}{\faEnvelope\ zhenxingjian1995@gmail.com} & Meta Platforms, Inc. \\
    \faPhone\ (XXX)-XXX-XXXX & 1550 121st Ave NE \\
    \href{https://zhenxingjian.github.io/homepage}{\faHome\ https://zhenxingjian.github.io/homepage} & Bellevue, WA, USA  \\
    \href{https://github.com/zhenxingjian}{\faGithub \ https://github.com/zhenxingjian} & 98005
\end{tabular}
}


\begin{resume}
\vspace{-2em}
\section{RESEARCH}
                \textbf{Deep Learning, Multi-modality, Computer Vision, Efficiency, and Medical Imaging}
%                My research interest is two-fold. On the algorithm side, I'm interested in structured data, such as covariance matrices of features and graphs. Those features are constrained by their own structure, i.e., Symmetric Positive Definite, and break most basic operations in Euclidean space. But the key benefit is that structured data are more robust to noise and contain more high-level information. On the application side, I would like to explore such data structures in medical imaging and multi-modality (such as image-text). In the medical imaging setting, structure information is sensitive and enables us to detect micro changes in the early stage of the disease. More importantly, since there naturally exists structures among modalities, we can use the structures to improve the multi-modality performance (e.g. VQA).                 
%                My research interest is about structured data or high-order features in Computer Vision. Since it's quite mature for Euclidean space deep learning, I would like to explore some different structured data or constrained data. The covariance matrices of the feature vectors can be viewed as the second-order features. Those high-order features are constrained by their own structure, i.e., Symmetric Positive Definite. Thus this will break the basic rules in Euclidean space. But those matrices are more robust to noise and contain more high-level information. I want to try some neural networks in Euclidean space, and extend them to those structured data. 
                
\vspace{-1em}
\section{EDUCATION} 
                {\bf University of Wisconsin-Madison}, Madison, WI, U.S. \hfill 08/2017-05/2023\\
                {\sl {Ph.D.}}, Department of Computer Science \hfill GPA: 3.89/4.0\\
%                {Advisor:} Prof. Vikas Singh\\
%                {Doctoral Minor:} Mathematics\\
                {\bf Tsinghua University}, Beijing, P.R. China. \hfill 08/2013-06/2017\\
                {\sl {B.E.}}, Department of Electronic Engineering \hfill GPA: 90.9/100

%\vspace{-2em}                
%\section{UNDER\\ REVIEWING}
%				\begin{enumerate}[noitemsep,wide=0pt,leftmargin=\dimexpr\labelwidth + 2\labelsep\relax]\itemsep -0.0pt
%				\setcounter{enumi}{7}
%%                \item \textbf{[\emph{In submission}]} ``Can Kernel Transfer Operators Help Flow based Generative Models?''\\
%%                            Zhichun Huang, Rudrasis Chakraborty, \textbf{Xingjian Zhen}, Vikas Singh.
%                \end{enumerate}

\vspace{-1em}
\section{WORK \\ EXPERIENCES}
				{\it Research Scientist}, \textbf{Meta}, Bellevue, WA, U.S. \hfill 02/2024-present
                \begin{itemize}\itemsep -2.2pt %reduce space between items
                \item[-] Suggested enhancements to the recommendation system, aimed at optimizing efficiency across an extensive dataset of over 2.8 billion real-world users
                \item[-] Proposed streamlining the pipeline within Facebook's internal platform, aiming to alleviate the workload of deep learning engineers
                 \end{itemize}
                 
                 \vspace{-1em}
				{\it Scientist I}, \textbf{Allen Institute for Brain Science}, Seattle, WA, U.S. \hfill 04/2023-02/2024
                \begin{itemize}\itemsep -2.2pt %reduce space between items
                \item[-] Utilized Variational Autoencoder (VAE) techniques to analyze both single-cell RNA and multimodal data, resulting in a remarkable 0.86 mean F1 score and 91\% accuracy across 138 classes
                \item[-] Pioneered the implementation of the innovative approach, named CELL and CELLBLAST, on MapMyCells platform, democratizing access for the wider scientific community                 
                \end{itemize}

				\vspace{-1em}

				{\it Research Scientist Intern}, \textbf{Meta}, Menlo Park, CA, U.S. \hfill 05/2022-09/2022
%                \textbf{Mentor:} Vijay Mahadevan, Zhuowen Tu, and Qi Dong
%                \textbf{Title:} Applied Scientist Intern
                \begin{itemize}\itemsep -2.2pt %reduce space between items
                \item[-] Introduced 4 novel techniques for communicating gradient information among multiple runs of deep learning recommendation models
                \item[-] Demonstrated significant improvements on the Criteo 1T benchmark, achieving a 1\% better test AUROC and reducing the generalization gap by 4\%
                 \end{itemize}

				\vspace{-1em}
				{\it Applied Scientist Intern}, \textbf{Amazon}, Pasadena, CA, U.S. \hfill 05/2021-08/2021
%                \textbf{Mentor:} Vijay Mahadevan, Zhuowen Tu, and Qi Dong
%                \textbf{Title:} Applied Scientist Intern
                \begin{itemize}\itemsep -2.2pt %reduce space between items
                \item[-] Proposed a transformer-based agent method designed to process pairwise input for co-detection
                \item[-] In our in-house dataset, our approach exhibited 6\% enhancement in F1 score over current SOTA
                \item[-] On the MS-COCO dataset, our method outperformed Deformable DETR by an impressive 4\%
                 \end{itemize}
                
                \vspace{-1em}
                {\it Applied Scientist Intern}, \textbf{Amazon}, Seattle, WA, U.S. \hfill 05/2020-09/2020
%                \textbf{Mentor:} Karen Hovsepian, and Mingwei Shen
%                \textbf{Title:} Applied Scientist Intern
                \begin{itemize}\itemsep -2.2pt %reduce space between items
                \item[-] Developed MTXNet for generating answers, textual, and visual explanations for TextVQA 
                \item[-] Got 1\% better accuracy, 7\% better textual explanation CIDEr, 2\% better visual explanation IoU
                 \item[-] Curated the TextVQA-X dataset from the TextVQA dataset with additional explanatory content
%                 \item[-] Published paper [4] in NAACL-MAI-Workshop 2021, together with oral presentation internally
                 \end{itemize}
                
                \vspace{-1em}  
                
                {\it Research Scientist Intern}, \textbf{DAMO Academy, Alibaba}, Beijing, P.R. China. \hfill 05/2019-09/2019
%                \textbf{Mentor:} Ying Chi
%                \textbf{Title:} Research Intern
                \begin{itemize}\itemsep -2.2pt %reduce space between items
                 % \item[-] Research intern in DAMO Academy Medical AI Algorithm Research, Alibaba
                 \item[-] Developed an automated anatomical labeling system for coronary arteries, employing CPR-GCN
                 \item[-] Utilized 3D CNN with BiLSTM to extract features from CT images along arterial branches
                 \item[-] Achieved a mean recall of 95.8\%, showcasing a 9\% improvement over the baseline performance
%                 \item[-] Published paper [8] with oral presentation in CVPR 2020 ($5.7\%$ acceptance rate)
                 \end{itemize}

\vspace{-1em}
\section{RESEARCH \\ EXPERIENCES}
				\textbf{Partial Distance Correlation (PDC) in Deep Learning and the Benefit}\hfill 12/2020-04/2022
                % Advisor: Prof. Vikas Singh, Dept. of Biostatistics and the Dept. of Computer Sciences at the University of Wisconsin-Madison
                \begin{itemize}\itemsep -2.2pt %reduce space between items
                \item[-] This work won {\textcolor{red}{Best Paper Award}} in ECCV 2022
                 \item[-] Introduced DC to enhance the robustness, getting a 9\% lower transferred attack rate with PGD
%                 \item[-] Used DC to disentangle the latent representation, and generated SOTA manipulate image on FFHD
                 \item[-] Utilized Partial DC to selectively remove information from one network to another, providing insights into the performance variations among different models
%                 \item[-] Published paper [3] in ECCV 2022 ($28\%$ acceptance rate)
                 \end{itemize}

                 \vspace{-1em} 
                \textbf{Certified Robustness Training via Propagating Gaussian Distribution}\hfill 01/2020-11/2020
                % Advisor: Prof. Vikas Singh, Dept. of Biostatistics and the Dept. of Computer Sciences at the University of Wisconsin-Madison
                \begin{itemize}\itemsep -2.2pt %reduce space between items
                 \item[-] Implemented the certifiable randomized smoothing robustness without sampling with $2\times$ faster
                 \item[-] Introduced a precise estimation method for Gaussian distribution to reduce computational overhead
                 \item[-] Achieved improved certified accuracy and a 5\% better robustness on ImageNet and Places365
%                 \item[-] Published paper [6] with oral presentation in CVPR 2021 ($4.6\%$ acceptance rate)
                 \end{itemize}

                 \vspace{-1em}  

                \textbf{Flow-based Generative Model for Non-Euclidean Data}\hfill 03/2019-12/2019
                % Advisor: Prof. Vikas Singh, Dept. of Biostatistics and the Dept. of Computer Sciences at the University of Wisconsin-Madison
                \begin{itemize}\itemsep -2.2pt %reduce space between items
                 \item[-] Proposed the integration of 3 invertible layers specifically designed for manifold-valued data
                 \item[-] Developed a two-stream GLOW architecture capable of transferring information between manifolds
                 \item[-] Successfully transferred Diffusion Tensor Imaging (DTI) data to corresponding Orientation Distribution Function (ODF) data while preserving verifiable group differences (with a $p$-value of $<$0.001).
%                 In reality, this can save scanning time from $35$ mins to $7-9$ mins
%				\item[-] Published paper [7] in AAAI 2021 ($21\%$ acceptance rate)
                 \end{itemize}

                 \vspace{-1em}  

%                \textbf{Point Cloud Completion}\hfill 09/2019-11/2019
%                % Advisor: Prof. Vikas Singh, Dept. of Biostatistics and the Dept. of Computer Sciences at the University of Wisconsin-Madison
%                \begin{itemize}\itemsep -2.2pt %reduce space between items
%                 \item[-] Used the encoder-decoder based network to roughly complete the point cloud
%                 \item[-] Utilized the nearest neighbor in the training dataset to extract local information
%                 \end{itemize}

                \textbf{Manifold Dilated CNN in Group Analysis of Alzheimer's Disease} \hfill 08/2018-02/2021
                % Advisor: Prof. Vikas Singh, Dept. of Biostatistics and the Dept. of Computer Sciences at the University of Wisconsin-Madison
                \begin{itemize}\itemsep -2.2pt %reduce space between items
%                 \item[-] Used TractSeg and Tractometry to compute the average representation along 50 fiber bundles
                 \item[-] Incorporated the SPD and $S^n$ manifolds into the DCNN model, enhancing the extraction of information from DTI and ODF data.
                 \item[-] Accelerated both training and testing processes by $5\times$, while maintaining a competitive number of parameters compared to the SOTA models
                 \item[-] Demonstrated statistically significant differences in 14 and 16 (out of 50) fiber bundles, as indicated by PiB-PET and gene mutation carriers, on the DIAN and WRAP datasets, respectively. 
%                 \item[-] Published paper [9] in ICCV 2019 ($25\%$ acceptance rate), and also extended to 2 AAIC abstracts
                 \end{itemize}

\vspace{-1em}                
\section{SELECTED PUBLICATIONS \\ \small{ 2 {\textcolor{red}{Oral}}\\ 7 Poster\\ 1 In Submission \\ 156 Citations \\ 200+ Stars GitHub}  }
                \begin{enumerate}[noitemsep,wide=0pt,leftmargin=\dimexpr\labelwidth + 2\labelsep\relax]\itemsep -0.0pt
                \item \textbf{[\emph{In Submission to Nature}]} ``A Multimodal Brain Cell Atlas and Community Resource of Alzheimer's Disease.''\\
                				\textbf{Zhen X.}, and et al.
%                \item \textbf{[\emph{In Submission}]} ``Variational Sampling of Temporal Trajectories.''\\
%                				Nazarovs J., Huang Z., \textbf{Zhen X.}, Pal S., Chakraborty R., 
%                            and Singh V.
%                \item \textbf{[\emph{In Submission}]} ``Non-Sequential Module Selection in Anytime Neural Networks.''\\
%                				Meng Z., \textbf{Zhen X.}, Ravi S., and Singh V.           
                \item \textbf{[\emph{ECCV \textcolor{red}{(Best Paper Award)}, 2022}]} ``On the Versatile Uses of Partial Distance Correlation in Deep Learning.''\\
                            \textbf{Zhen X.}, Meng Z., Chakraborty R., and Singh V.
                \item \textbf{[\emph{NAACL-MAI, 2021}]} ``A First Look: Towards Explainable TextVQA Models via Visual and Textual Explanations.''\\
                            \textbf{Zhen X.}$^*$, Rao V.N.$^*$, Hovsepian K., and Shen M.
                \item \textbf{[\emph{AAIC, 2021}]} ``Altered Structural Connectivity Detected with Dilated Convolutional Neural Network Analysis in the DIAN study and the Wisconsin Registry for Alzheimer's Prevention.''\\
                            \textbf{Zhen X.}, Chakraborty R., Vogt N., Wang R., Yang K.L., Adluru N., Gordon B., Benzinger T., Mckay N., Betthauser T., Johnson S.C., Singh V., and Bendlin B.B.
                \item \textbf{[\emph{CVPR \textcolor{red}{(Oral)}, 2021}]} ``Simpler Certified Radius Maximization by Propagating Covariances.''\\
                            \textbf{Zhen X.}, Chakraborty R., and Singh V. 
                \item \textbf{[\emph{AAAI, 2021}]} ``Flow-based Generative Models for Learning Manifold to Manifold Mappings.''\\
                            \textbf{Zhen X.}, Chakraborty R., Yang L., and Singh V. 
                \item \textbf{[\emph{CVPR \textcolor{red}{(Oral)}, 2020}]} ``CPR-GCN: Conditional Partial-Residual Graph Convolutional Network in Automated Anatomical Labeling of Coronary Arteries.''\\
                            \textbf{Zhen X.}$^*$, Yang H.$^*$, Chi Y., Zhang L., and Hua X.S.   
                \item \textbf{[\emph{ICCV, 2019}]} ``Dilated Convolutional Neural Networks for Sequential Manifold-valued Data.'' \\
                             \textbf{Zhen X.}$^*$, Chakraborty R.$^*$, Vogt N., Bendlin B.B., and Singh V. 
                \item \textbf{[\emph{AAIC, 2019}]} ``Sequential Deep Learning Algorithms Show Structural Connectivity Differences By Amyloid Status.''\\
                             \textbf{Zhen X.}, Chakraborty C., Vogt N., Hwang S.J., Johnson S.C., Bendlin B.B., and Singh V. 
                \item \textbf{[\emph{NeurIPS, 2018}]} ``A Statistical Recurrent Model on the Manifold of Symmetric Positive Definite Matrices.''\\
                             Chakraborty R., \textbf{Zhen X.}$^*$, Yang C.H.$^*$, Banerjee M., Archer D., Vaillancourt D., Singh V., and Vemuri B.C.
%                \item 
                \end{enumerate}
\vspace{-1em}
\section{COMPUTER \\ SKILLS} 
                Deep learning framework: PyTorch, TensorFlow\\
                Languages: Python, \LaTeX

%                 \vspace{-1em}  
%
%                 \textbf{Statistical Recurrent Model on the SPD Manifold} \hfill 01/2018-05/2018
%                % Advisor: Prof. Vikas Singh, Dept. of Biostatistics and the Dept. of Computer Sciences at the University of Wisconsin-Madison
%                \begin{itemize}\itemsep -2.2pt %reduce space between items
%                 \item[-] Defined the ``$+/\ \times$'' operators in the manifold space using group operators and wFM
%                 \item[-] Reduced the number of parameters of the video classification model $100\times$
%                 \item[-] Achieved SoTA accuracy, $78\%$, on UCF11 dataset
%                 \item[-] Published paper [8] in NeurIPS 2018 ($21\%$ acceptance rate)
%                 \end{itemize}

%                 \vspace{-1em}  

%                 \textbf{Correlations for Image-Text Pair in Latent Space} \hfill 09/2017-01/2018
%                \begin{itemize}\itemsep -2.2pt %reduce space between items
%                 \item[-] Applied the pre-trained CNN as the feature extractor from the image side 
%                 \item[-] Applied the word2vec method on the text as the representative of text
%                 \item[-] Used t-SNE to minimize the KL divergence between latent space and the Image-Text pair
%                 \item[-] Got meaningful results on the local dataset gathered from Reddit
%                 \end{itemize}

%                 \textbf{Form Line Detection in the Picture} \hfill 12/2016-06/2017
%                % Advisor: Prof. Liangrui Peng, Dept. of Electronic Engineering, Tsinghua Univ.
%                \begin{itemize}\itemsep -2.2pt %reduce space between items
%                 \item[-] Developed a system that detects and recognizes the form lines in pictures
%                 \item[-] Used the bidirectional RNN method to achieve the state of the art, with MXNet as the core of the deep-learning system 
%                 \item[-] Tested in multiple dataset such as the NIST Special Database 2 and got $99.0\%$ accuracy rate
%                 \end{itemize}

%
%\vspace{-1em}
%\section{REVIEWER \\ SERVICES}
%                2020: ECCV, MICCAI, NeurIPS,\\ 2021: AAAI, ICLR, CVPR, IJCAI, ICML, MICCAI, ICCV, NeurIPS

\end{resume}
\end{document}

